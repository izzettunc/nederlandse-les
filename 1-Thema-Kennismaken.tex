\documentclass[a4paper,14pt]{extarticle}
\usepackage{paracol}
\usepackage{tabularx}
\usepackage[a4paper, total={7in, 10.5in}]{geometry}
\newcommand{\optional}[1]{\underline{\textbf{(#1)}}\footnote{Optional}}
\newcommand{\note}[2]{\underline{\textbf{#1}}\footnote{#2}}
\newcommand{\attention}[1]{\underline{\textbf{!! #1}}}
\newcommand{\emp}[1]{\underline{\textbf{#1}}}
\begin{document}
\section{Kennismaken}
\begin{paracol}{2}
\emp{Asking for a name} \\ \\
Wat is jouw naam? \\
- Mijn naam is Izzet. \\ \\
Wie ben jij ? \\
- Ik ben Izzet. \\ \\
Hoe heet jij? \\
- Ik heet Izzet. \\ \\
Wat jouw voornaam? \\
- Mijn voornaam is Izzet. \\ \\
Wat jouw achternaam? \\
- Mijn \note{achter}{After}naam is Tunc. \\ \\
\emp{Asking for where you come from,} \\
\emp{live or work} \\ \\
Waar kom jij vandaan ? / Uit welk land kom jij? \\
-Ik kom uit Turkije. \\ \\
Waar komt Mike/Farah vandaan? \\
- Hij/Zij komt uit Zuid-Korea/England. \\ \\
Waar woon jij? \\
- Ik woon in Rotterdam. / Ik woon in Otto Verdoornplaats. \\ \\
Waar werk jij? \\
- Ik werk bij Rabobank. / Ik werk in een bank. \\ \\
Hoe lang ben je in Nederlands? \\
- Ik been negen maanden/twee jaar in Nederland. \\ \\
\switchcolumn
\emp{Asking for language related things} \\ \\
Welke talen spreek jij? \\
- Ik spreek Engels, Turks en een beetje Nederlands. \\ \\
Ik leer Nederlands. En jij? \\
- Ik leer \note{ook}{Too/Also} Nederlands. \\ \\
\emp{Asking for age related things} \\ \\
Wanneer ben jij jarig? \\
- Ik ben jarig op twee Agustus. \\ \\
Hoe oud ben je? \\
- Ik ben 24(vierentwintig) jaar \optional{oud} \\ \\
\emp{Pronunciation} \\ \\
\begin{tabularx}{275pt}{ p{110pt} p{165pt} }
 \hline
 Dutch spelling & Turkish pronounciation \\
 \hline
- ui & - aü \\
- oe & - u \\
- eu & - öü \\
- ou/au & - au \\
- ij/ei & - ey \\
- aa & - aa \\
- a & - a \\
- ee & - ey \\
- e & - e \\
- ie & - ii \\
- i & - i \\
- oo & - oo \\
- o & - o \\
- uu & - üü \\
- u & - ü \\
\end{tabularx} \\ \\ \\
\emp{Exception (schwa)} \\ \\
w\emp{e}/z\emp{e}/j\emp{e}/lekk\emp{e}r = vı/zı/yı/lekkır \\
moeil\emp{ij}k = muilık
\end{paracol}
\newpage
\subsection{Grammatica}
\subsubsection{Lidwoorden (Articles)}
\begin{paracol}{2}
de (the) de tafel \\
het (the) het boek \\
\attention{Plural nouns always have "de"} \\
de tafel \\
de boeken \\
\switchcolumn
Indefinite article is always "een" \\
e.g.\\
een (a/an) een tafel/een appel \\\\
\end{paracol}
\subsubsection{Personswoorden (Personal pronouns)}
\begin{paracol}{2}
   \hfill \\
\begin{tabularx}{200pt}{ p{100pt} p{100pt} }
 \hline
 \textbf{Enkelvoud} & \textbf{Single} \\
 \hline
 Ik & I \\ 
 Jij/Je/\note{U}{Formal} & You \\  
 Hij & He \\ 
 Zij/Ze & She \\ 
 Het/Hij & It \\
 \hline
 \textbf{Meervoud} & \textbf{Plural}\\
 \hline
 Wij/We & We\\
 Jullie & You(Plural)\\
 Zij/Ze & They\\
\end{tabularx}
\switchcolumn
\attention{Rule:} \\ \\
The rule of using het or hij when you want to refer to things is: \\
- Use het if the thing that you want to refer is a het word. \\
- Use hij if the thing that you want to refer is a de word. \\ \\
\attention{e.g. :} \\ \\
\emp{De appel} is lekker. - \emp{Hij} is lekker. \\
\emp{Het brood} is lekker. - \emp{Het} is lekker.
\end{paracol}
\rule{\textwidth}{1pt}
\begin{paracol}{2}
\hfill \\
Waar woont \emp{Pieter}? \\
- \emp{Hij} woont in de Dorpstraat \\ \\
Waar wonen \emp{Julia en Brahim}? \\
- \emp{Zij} wonen in Utrecht. \\ \\
Waar werkt \emp{Anna}? \\
- \emp{Zij} werkt in een hotel. \\ \\
Waar zijn \emp{Henk en jij}? \\
- \emp{Wij} zijn in een restaurant. \\ \\
Waar werken \emp{mijn man en ik}? \\
- \emp{Jullie} werken op school.
\switchcolumn
\hfill \\
Waar sport \emp{jij}? \\
- \emp{Ik} sport in een sporthal \\ \\
Waar ben \emp{ik}? \\
- \emp{Je} bent in de klas/\emp{Je} bent thuis. \\ \\
\attention{Pronoun stress} \\ \\
\begin{tabularx}{200pt}{ p{100pt} p{100pt} }
\hline
\emp{Stressed} & \emp{Unstressed} \\
\hline
Jij & Je \\
Zij & Ze \\
Wij & We 
\end{tabularx} \\ \\ \\
Used to emphasize the pronoun \\
e.g.
Jij used to specifically mention something related to especially you.
\end{paracol}
\newpage
\subsubsection{Persoonswoorden en werkwoorden (Personal pronouns and verbs)}
\begin{center}
\begin{tabularx}{400pt}{ c c c }
 \hline
 \multicolumn{3}{ c }{Werkwoorden:Werken(Regelmatig)} \\
 \hline
 \multicolumn{3}{ c }{Verb:Work(Regular Verb)} \\
 \hline
 Ik & Werk\_ & No 'en'\\ 
 Jij/Je/U & Werk\underline{t} & No 'en' but 't' \\  
 Hij/Zij/Ze/Het & Werk\underline{t} & No 'en' but 't' \\ 
 Wij/We/Jullie/Zij/Ze & Werken & Same as the verb \\
 \multicolumn{3}{ c }{\attention{Werk jij/je ? (NO 't'!)} but \attention{Werkt u? (THERE IS T)}} \\
 & & \\
    & \underline{Wonen}(to live) & \underline{Spreken}(to speak)\\
 Ik & W\underline{oo}n & Spr\underline{ee}k\\ 
 Jij/Je/U & W\underline{oo}n\underline{t} & Spr\underline{ee}k\underline{t} \\  
 Hij/Zij/Ze/Het & W\underline{oo}n\underline{t} & Spr\underline{ee}k\underline{t} \\ 
 Wij/We/Jullie/Zij/Ze & Wonen & Spreken \\
  \multicolumn{3}{ l }{\attention{Verbs with a single vowel followed by a single consonant}} \\
  \multicolumn{3}{ l }{\attention{ in the first syllable get an extra vowel with conjugation}} \\
 & & \\
    & \underline{Wassen}(to wash) & \underline{Zeggen}(to say)\\
 Ik & Was\_\_ & Zeg\_\_\\ 
 Jij/Je/U & Was\_\underline{t} & Zeg\_\underline{t} \\  
 Hij/Zij/Ze/Het & Was\_\underline{t} & Zeg\_\underline{t} \\ 
 Wij/We/Jullie/Zij/Ze & Wassen & Zeggen \\
 \multicolumn{3}{ l }{\attention{Verbs with a single vowel followed by double consonant}} \\
 \multicolumn{3}{ l }{\attention{drops one consonant with conjugation}} \\
\end{tabularx}
\end{center}
\begin{center}
\begin{tabularx}{350pt}{ c l l }
 \hline
 \multicolumn{3}{ c }{Werkwoorden:Zijn\&Hebben(Onregelmatig )} \\
 \hline
 \multicolumn{3}{ c }{Verb:To be\&To have(Irregular Verb)} \\
 \hline
     & \underline{Zijn}(to be) & \underline{Hebben}(to have)\\
 Ik & Ben & Heb \\ 
 Jij/Je/U & Bent & Hebt \\  
 Hij/Zij/Ze/Het & Is & Heeft \\ 
 Wij/We/Jullie/Zij/Ze & Zijn & Hebben \\
\end{tabularx}
\end{center}
\subsubsection{Zinnen(Sentences)}
\begin{center}
\begin{tabularx}{250pt}{ l l }
 Hij leest (een boek) & He reads a book. \\ 
 Leest hij (een boek) & Does he read a book? \\  
 Hij leest niet & He does not read. \\
 Hij leest geen boek & He does not not read a book. \\ 
\end{tabularx}
\end{center}
\attention{A common sentence starts with a personal pronoun(or a name)}\\
\attention{followed by a verb and there is never anything in between them.}\\
\attention{In questions the word order changes to first verb then noun.}
\newpage
\subsection{Getallen}
\begin{paracol}{2}
\hfill \\
0 nul \\
1 een \\
1.5 anderhalf \\
2 twee \\
3 drie \\
4 vier \\
4.5 vierenhalf \\
5 vijf \\
6 zes \\
7 zeven \\
8 acht \\
9 negen \\
10 tien \\
11 elf \\
12 twaalf \\
13 dertien \\
14 veertien \\
15 vijftien \\
16 zestien \\
17 zeventien \\
18 achtien \\
19 negentien \\
20 twintig \\
\switchcolumn
\hfill \\
21 eenentwintig \\
22 tweeentwingtig \\
23 drieentwintig \\
24 vierentwintig \\
25 vijfentwintig \\
30 dertig \\
40 viertig \\
50 vijftig \\
60 zestig \\
70 zeventig \\
80 tachtig \\
90 negentig \\
100 honderd \\
101 honderdeen \\
200 tweehonderd \\
1000 duizend \\
1001 duizendeen \\
1.000.000 een miljoen \\
1.000.000.000 een miljard \\
\end{paracol}
\end{document}