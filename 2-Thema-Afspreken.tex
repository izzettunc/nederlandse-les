\documentclass[a4paper,14pt]{extarticle}
\usepackage{paracol}
\usepackage{tabularx}
\usepackage[a4paper, total={7in, 10.5in}]{geometry}
\newcommand{\optional}[1]{\underline{\textbf{(#1)}}\footnote{Optional}}
\newcommand{\note}[2]{\underline{\textbf{#1}}\footnote{#2}}
\newcommand{\attention}[1]{\underline{\textbf{!! #1}}}
\newcommand{\emp}[1]{\underline{\textbf{#1}}}
\begin{document}
\section{Afspreken}
\subsection{Dag en tijd}
\begin{paracol}{2}
\paragraph{Maanden}
\hfill \\
1 Januari \\
2 Februari \\
3 Maart \\
4 April \\
5 Mei \\
6 Juni \\
7 Juli \\
8 Agustus \\
9 September \\
10 Oktober \\
11 November \\
12 December \\
\paragraph{Seizoenen}
\hfill \\
1 De Winter \\
2 De Lente \\
3 De Zomer \\
4 De Herfts \\
\paragraph{Tijden}
\hfill \\
de morgen  - the morning \\
de ochtend - the morning \\
de middag - the afternoon \\
de avond - the evening \\
de nacht - the night \\
\\
’s morgens - in the morning \\
’s ochtends - in the morning \\
’s middags - in the afternoon \\
’s avonds - in the evening \\
’s nachts - in the night \\
\switchcolumn
\paragraph{Dag van de week}
\hfill \\
1 Maandag \\
2 Dinsdag \\
3 Woensdag \\
4 Donderdag \\
5 Vrijdag \\
6 Zaterdag \\
7 Zondag \\
\paragraph{De klok}
\hfill \\
10.00 Het is tien \emp{uur} \\
10.05 Het is vijf \emp{over tien} \\
10.10 Het is tien \emp{over tien} \\
10.15 Het is \emp{kwart over tien} \\
10.20 Het is tien \emp{voor half elf} \\
\phantom{10.20 }Het is twintig \emp{over tien} \\
10.25 Het is vijf \emp{voor half elf} \\
10.30 Het is \emp{half elf} \\
10.35 Het is vijf \emp{over half elf} \\
10.40 Het is tien \emp{over half elf} \\
\phantom{10.20 }Het is twintig \emp{voor} elf \\
10.45 Het is \emp{kwart voor} elf \\
10.50 Het is tien \emp{voor} elf \\
10.55 Het is vijf \emp{voor} elf \\
11.00 Het is elf \emp{uur} \\
\end{paracol}
\newpage
\subsection{Grammatica}
\subsubsection{Inversie}
\begin{paracol}{2}
\attention{RULE:} \\
\textbf{Normal:} Subject + verb + (time) + rest  \\
\textbf{Inverted:} Time or place + verb + subject + rest  \\
\\
\textbf{Normaal:} \emp{Zij} werkt \emp{op donderdag} in een restaurant. \\
\textbf{Inversie:} \emp{Op donderdag} werkt \emp{zij} in een restaurant. \\
\\
\textbf{Normaal:} \emp{Het} regent \emp{morgen} in Rotterdam. \\
\textbf{Inversie:} \emp{Morgen} regent \emp{het} in Rotterdam. \\
\textbf{Inversie:} \emp{In Rotterdam} regent \emp{het} morgen. \\
\switchcolumn
\attention{Note:} \\
!) naar = naartoe \\
- Elke week gaat Pieter naat het cafe. \\
- Morgen ga ik nergens naartoe. \\
\emp{Except "naar" used if there is a destin-} \\
\emp{ation} \\
\end{paracol}
\subsubsection{Negatief}
\emp{"niet"} and \emp{"geen"} used to make a negative sentence. \\
\emp{e.g. :} \\
Ga je naar school? - Nee, ik ga \emp{niet} naar school. \\
Heb jij kinderen? - Nee, ik heb \emp{geen} kinderen. \\
Waaroom koop je de trui \emp{niet}? - De trui is duur dus ik koop hem {niet}. \\
Waaroom willen jullie \emp{geen} afspraak maken? - Want, wij hebben \emp{geen} tijd. \\
\\
\attention{Rule:} \\
\emp{"geen"} is used for nouns and for the rest it is \emp{"niet"}.
\subsubsection{Houden van, graag, en lust}
\emp{houden van} \\ \\
\emp{"houden van"} is used to indicate you like something and it \emp{is} a verb. \\
\emp{e.g. :} \\
- \emp{Hou} je niet \emp{van} werken? - Ja, Ik \emp{hou} niet \emp{van} werken. \\
- \emp{Hou} je \emp{van} pasta? - Ja, Ik \emp{hou van} pasta. \\
- \emp{Hou} je \emp{van} kinderen hebben? - Ja, Ik \emp{hou van} kinderen hebben. \\
\attention{While "Hou" is informal, "houdt" is formal.}
\newpage
\hfill \\
\emp{graag} \\ \\
\emp{"graag"} is used to indicate you like something and it \emp{is not} a verb, therefore needs a verb to make sense. \\
\emp{e.g. :} \\
- Nee, ik zwem niet \emp{graag}. \\
- Ik loop \emp{graag}. \\ \\
\attention{If there is no verb, you use ‘hebben’, to have, with graag.} \\
- Ik \emp{heb graag} kinderen. (I like children)\\ \\
\emp{lust} \\ \\
\emp{"lust"} is used to indicate you \emp{like to eat} something and it \emp{is} a verb\\
\emp{e.g. :} \\
- Ik \emp{lust} pasta. \\
- Ik \emp{lust} geen koffie. \\
\subsubsection{Vraag stellen}
\emp{De vraagzin} \\ \\
\attention{Rule:} \\
Verb + subject + rest of the sentence \\
\emp{e.g. :} \\
- \emp{Kan} ik een afspraak maken? \\
- \emp{Werkt} hij elke dag? \\ 
- \emp{Woont} hij in Utrecht?\\ \\
\emp{De vraagwoorden} 
\begin{center}
\begin{tabularx}{\textwidth}{ l l p{350pt} }
 \hline
 Engels & Nederlands & Voorbeld \\
 \hline
 Who & Wie & \emp{Wie} ben je? - Ik ben Sara. \\
 What & Wat & \emp{Wat} zoek je? - Ik zoek mijn tas. \\
 Where & Waar & \emp{Waar} woon je? - Ik woon in Rotterdam. \\
 When & Wanneer & \emp{Wanneer} ga je naar kantoor? - Ik ga naar kantoor elke donderdag. \\
 How & Hoe & \emp{Hoe} kom je naar kantoor? - Ik kom met de trein naar kantoor. \\
 Which & Welk(e) & Op \emp{welke} dag heb je les? - Ik heb les op donderdag. \\
 Why & Waarom & \emp{Waarom} ga je niet naar de les? - Ik ben ziek.
\end{tabularx}
\end{center}
\newpage
\subsubsection{Modale werkwoorden}
A modal verb usually requires another verb. In some cases it is possible to drop the other verb, when this other verb is "hebben" or "gaan" but this is not possible for "zullen". Also the words in the paranthesis are gramatically correct to use but they are not really used in daily speaking.
\begin{center}
\begin{tabularx}{\textwidth}{ l l p{250pt} }
 \hline
 Engels & Nederlands & Voorbeld \\
 \hline
 Will/Shall & Zullen & \emp{Zullen} wij een afspraak maken? \\
 Want & Willen & Ik \emp{wil} een nieuwe trui kopen. \newline \emp{Wil} je naar huis (gaan)? \\
 Can  & Kunnen & Hij \emp{kan} op vrijdag niet werken want hij is ziek. \newline \emp{Kan} ik naar huis (gaan)? \\
 Must/Have to & Moeten & We \emp{moeten} op tijd op school komen. \newline Ik \emp{moet} naar huis (gaan). \\
 May/Being allowed to & Mogen & In het restaurant \emp{mag} ik niet roken. \newline \emp{Mag} ik een biertje (hebben)? \\
 Don't have to/Don't need to & Hoeven & Moet jij morgen werken? \newline - Nee, ik \emp{hoef} morgen \emp{niet} te werken. \newline Moet ik het formulier invullen? \newline - Nee, je \emp{hoeft} het formulier \emp{niet} in \emp{te} vullen. \newline Wil je een koekje (hebben)? \newline -Nee, ik \emp{hoef geen} koekje (te hebben). \\
\end{tabularx}
\end{center}
\attention{Note:} \\
Hoeven is always used in combination with NIET or GEEN and depending on the sentence type also with TE + OTHER VERB. It is used as the opposite of MOETEN.
\\
\attention{Rule:} \\
HOEVEN + NIET + TE + VERB \\
HOEVEN + GEEN + NOUN (+ TE + VERB)
\subsubsection{De hoofdzin met twee werkwoorden}
\textbf{Normal:} Subject + verb + (time) + rest  \\
\textbf{Two verb:} Subject + verb + (time) + rest + other verb \\
\\
\textbf{Normaal:} Peter \emp{maakt} een afspraak.\\
\textbf{Twee werkwoorden:} Peter \emp{wil} een afspraak \emp{maken}. \\
\hfill \\
\emp{e.g. :} \\
- Ik \emp{ga} morgen de hele dag \emp{sporten}. \\
- \emp{Wil} je nieuwe klerene \emp{kopen}? \\
- Nee, Ik \emp{wil} geen nieuwe kleren \emp{kopen}. \\
\end{document}