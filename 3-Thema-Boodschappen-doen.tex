\documentclass[a4paper,14pt]{extarticle}
\usepackage{paracol}
\usepackage{tabularx}
\usepackage[a4paper, total={7in, 10.5in}]{geometry}
\newcommand{\optional}[1]{\underline{\textbf{(#1)}}\footnote{Optional}}
\newcommand{\note}[2]{\underline{\textbf{#1}}\footnote{#2}}
\newcommand{\attention}[1]{\underline{\textbf{!! #1}}}
\newcommand{\emp}[1]{\underline{\textbf{#1}}}
\begin{document}
\section{Boodschappen doen}
\subsection{Eten en drinken}
\begin{paracol}{2}
\begin{tabularx}{200pt}{ p{100pt} p{100pt} }
 \hline
 Engels & Nederlands\\
 \hline
 Milk & Melk \\
 Cucumber & Komkommer \\
 Cheese & Kaas \\
 Egg/Eggs & Ei/Eiren \\
 Yoghurt & Yoghurt \\
 Tomato & Tomaat \\
 Carrot & Wortels \\
 Strawberry & Aardbei \\
 Lettuce & Sla \\
 Apple & Appel \\
 Banana & Banaan \\
 Pear & Peer \\
 Kiwi & Kiwi \\
 Green beans & Sperziebonen \\
 Custard & Vla \\
 Meat & Vlees \\
 Bread & Brood \\
 Vegetables & Groenten \\
 Fruit & Fruit \\
 Chicken & Kip \\
 Tea & Thee \\
 Cookies & Koekjes \\
 Pie & Taart \\
 Soft drink & Frisdrank \\
 Chips & Chips \\
 Jam & Jam \\
 Baker & Bakker \\
 Bakery & Bakkerij \\
 Butcher & Slager \\
 Butchery & Slagerij \\
 Breakfast & Ontbijt \\
 Lunch & Lunch/Middageten \\
 Dinner & Avondeten \\
 Snack & Tussendoortjes \\
 Peanut butter & Pindakaas \\
\end{tabularx}
\switchcolumn
\begin{tabularx}{200pt}{ p{100pt} p{100pt} }
 \hline
 Engels & Nederlands\\
 \hline
 A jar of peanut butter & Een \emp{pot} pindakaas \\
 A carton of milk & Een \emp{pak} melk \\
 A bag of potatoes & Een \emp{zak} aardappels \\
 A box of tea & Een \emp{doos} thee \\
 A can of tomato paste & Een \emp{blikje} tomatenpuree \\
 A pack of of coffee & Een \emp{pak} koffie \\
\end{tabularx}
\hfill \\ \\ \\
\emp{Nodig} \\ \\
(Hebben) nodig used to indicate that you require/need \emp{something}. \\
\emp{e.g. : } \\
- Ik heb ook kaas \emp{nodig}. \\
- Dat is niet \emp{nodig}. \\
- Hij heeft hulp \emp{nodig}. \\ \\
\emp {Zin} \\ \\
(Hebben) zin used to indicate that you feel like it/in the mood of or not. \\
\emp{e.g. : } \\
Ga je mee naar Amsterdam ? - Nee, ik \emp{heb geen zin}. \\
\emp{Heb} je \emp{zin} om te dansen ? - Ja, ik \emp{heb zin} om te dansen. \\
\end{paracol}
\newpage
\subsection{Grammatica}
\subsubsection{De lidwoorden}
"De" and "Het" are two articles used for nouns just like "the".
\begin{paracol}{2}
\hfill \\
\emp{e.g. : } \\
- De appel is groot. \\
- De broden zijn goedkoop.
\switchcolumn
\hfill \\ \\
- Het brood is goedkoop. \\
- Het ei is duur.
\end{paracol}
\hfill \\
\attention{Rule:} \\
There is no specific rule to find if a noun is a het word or de word. But there are some indicators. \\ \\
\emp{e.g. : } \\
- All plurals are \emp{de} words. \\
- Most fruit and vegetables are \emp{de} words. \\
- Most fish and drinks are \emp{de} words. \\
- All words that starts with \emp{"ge"} are \emp{het} words. \\
- All words that starts with \emp{"ver"} are \emp{de} words. 
\subsubsection{'en' en 'of'}
En(as in and) and of(as in or) are used to connect to sentences together.\\ \\
\begin{tabularx}{\textwidth}{ p{0.25\textwidth} p{0.025\textwidth} p{0.25\textwidth} p{0.4\textwidth} }
 \hline
 Hoofdzin1 &  & Hoofdzin2 & Combinatie\\
 \hline
 Ik wil een kilo appels. & \emp{en} & Ik wil een pond bananen. & Ik wil een kilo appels \emp{en} een pond bananen. \\
 Julia koopt twee peren. & \emp{of} & Julia koopt drie peren. & Julia koopt twee \emp{of} drie peren. \\
\end{tabularx}
\subsubsection{Het meervoud}
There are three rules to turn a singular noun into plural one.\\ \\
\emp{a) With -s} \\ \\
If a word ends with below sound, we put an \emp{(s)} at the end of the word to turn it into plural. \\ \\
\emp{Sounds:} -el, -er, -en, -em, -je \\ \\
\begin{tabularx}{\textwidth}{ p{0.5\textwidth} p{0.5\textwidth} }
 \hline
 Enkelvoud(Singular) & Meervoud(Plural)\\
 \hline
 De wink\emp{(el)} & De winkel\emp{s} \\
 De numm\emp{(er)} & De nummer\emp{s} \\
 Het tas\emp{(je)} & De tasje\emp{s}
\end{tabularx}
\newpage
\hfill \\
\emp{b) With -'s} \\ \\
If a word ends with below sound, we put an \emp{('s)} at the end of the word to turn it into plural. \\ \\
\emp{Sounds:} a, o, u, i, y \\ \\
\begin{tabularx}{\textwidth}{ p{0.5\textwidth} p{0.5\textwidth} }
 \hline
 Enkelvoud(Singular) & Meervoud(Plural)\\
 \hline
 De op\emp{(a)} & De opa\emp{'s} \\
 De aut\emp{(o)} & De auto\emp{'s} \\
 De tax\emp{(i}) & De taxi\emp{'s}
\end{tabularx} \\ \\ \\
\emp{c) With -en} \\ \\
Rest of the words can be turned into plural by adding \emp{(en)} at the end of the word. \\ \\
\begin{tabularx}{\textwidth}{ p{0.5\textwidth} p{0.5\textwidth} }
 \hline
 Enkelvoud(Singular) & Meervoud(Plural)\\
 \hline
 De aardb\emp{(ei)} & De aardbei\emp{en} \\
 De jas & De jass\emp{en} \\
 De tr\emp{(ui)} & De trui\emp{en} \\
 Het brood & De brod\emp{en}
\end{tabularx} \\ \\ \\
\emp{d) Exceptions} \\ \\
As always there are few exceptions to these rules, some nouns are irregular and doesn't obey any rules and for some words, some sounds could be shortened or some additional letters can be added. \\ \\
\begin{tabularx}{\textwidth}{ p{0.5\textwidth} p{0.5\textwidth} }
 \hline
 Enkelvoud(Singular) & Meervoud(Plural)\\
 \hline
 De man & De ma\emp{nn}en \\
 De zus & De zu\emp{ss}en \\
 De vr\emp{aa}g & De vr\emp{a}gen \\
 De sch\emp{oo}l & De sch\emp{o}len \\
 Het ei & De eiren \\
 De koi & De koien \\
\end{tabularx}
\newpage
\subsubsection{De voorzetsels van plaats}
These are the words that used to indicate place of something. \\ \\
\begin{tabularx}{\textwidth}{ p{0.25\textwidth} p{0.25\textwidth} p{0.50\textwidth} }
 \hline
 Engels & Nederlands & Voorbeeld\\
 \hline
 On & Op & De kat staat \note{op de}{With article!} tafel. \\
 Under & Onder & De kat staat \emp{onder} de tafel.\\
 In front of & Voor & De kat staat \emp{voor} de tafel en de stoel.\\
 Behind & Achter & De kat staat \emp{achter} de tafel en de stoel\\
 Between & Tussen & De tafel staat \emp{tussen} de stoelen.\\
 Around & om/rond & De stoelen staan \emp{om/rond} de tafel.\\
 Next to & naast & De kat staat \emp{naast} de tafel.\\
 By & Bij & De kaat staat \emp{bij} de tafel.\\
 In & In & De kaat staat \emp{in} de doos.\\
 (Hanged)On & Aan & De klok hangt \emp{aan} de muur.\\
 Against & Tegen & De bank staat \emp{tegen} de muur.\\
 Above(not on) & Boven & De lamp hangt \emp{boven} de tafel.\\
 At & Op & Ik ben \note{op}{No article!} school/kantoor.
\end{tabularx} \\ 
\hfill \\
\begin{paracol}{2}
\subsubsection{'er' bij een getal}
Er is used as a placeholder to not repeat a word. \\
\emp{e.g. : } \\ \\
Ik heb drie \emp{appels}. \\
- Ik heb \emp{er} drie. \\ \\
Hoeveel \emp{flessen water} willen we? \\
- We willen \emp{er} twee. \\ \\ 
Wilt u vijf \emp{bananen}? \\ 
- Nee, ik will \emp{er} meer/een. \\ \\
Hoeveel \emp{kindren} heb jij? \\
- Ik heb \emp{er} geen. \\ \\
Wie is \emp{er} aan de \note{beurt}{Turn}? \\
Whose turn is it? \\
\switchcolumn
\hfill \\
\attention{Note:} \\ \\
\emp{Plaats:} voor - achter \\
\emp{Place:} front - behind \\ \\
\emp{Tijd:} voor - na \\
\emp{Time:} before - after \\ \\
\emp{Maar} \\
- At the following sentence it has no meaning, it just makes the sentence more friendly and polite. \\ \\
Zegt u het \emp{maar}? \\ \\
- In this sentence it means only. \\ \\
De druiven kosten vandaag \emp{maar} 1.50 euro per kilo.
\end{paracol}
\end{document}