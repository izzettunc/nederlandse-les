\documentclass[a4paper,14pt]{extarticle}
\usepackage{paracol}
\usepackage{tabularx}
\usepackage[a4paper, total={7in, 10.5in}]{geometry}
\newcommand{\optional}[1]{\underline{\textbf{(#1)}}\footnote{Optional}}
\newcommand{\note}[2]{\underline{\textbf{#1}}\footnote{#2}}
\newcommand{\attention}[1]{\underline{\textbf{!! #1}}}
\newcommand{\emp}[1]{\underline{\textbf{#1}}}
\begin{document}
\section{In de Winkel}
Het gaat over iets kopen en betalen.
\subsection{Artikelen kopen en ruilen}
\begin{paracol}{2}
\begin{tabularx}{200pt}{ p{100pt} p{100pt} }
 \hline
 Engels & Nederlands \\
 \hline
 To return & Terugbrengen \\
 To exchange & Ruilen \\
 To fit & Passen \\
 Suit & Pak \\
 Dress & Jurk \\
 Gloves & Handschoenen \\
 Shoes & Schoenen \\
 Skirt & Rok \\
 Beanie & Muts \\
 Shirt & Overhemd \\
 Tie & Stropdas \\
 Socks & Sokken \\
 Boots & Laarzen \\
 Shawl & Sjaal \\
 Raincoat & Regenpak \\
 Clothing & Kleding \\
 Clothes & Kleren \\
 Pants & Broek \\
 Jeans & Spijkerbroek \\
 Hat & Hoed \\
 Cap & Pet \\
 Uniform & Uniform \\
 Coat & Jas \\
 Shorter dress & Jurkje \\
 Sleveeless Undershirt & Hemd \\
 Fitting room & Paskamer \\
 Ankle boots & Enkellaarsjes \\
 Bracelet & Armband \\
 Necklace & Ketting \\
 Watch & Horloge \\
 Ring & Ring \\
 Shorts & Korte broek \\
 Jewelry & Sieraden \\
 Earrings & Oorbellen \\
\end{tabularx}
    \switchcolumn
Het gaat over \dots = It is about \dots \\ \\
\attention{Pronounciation} \\
-ng is always pronounced with soft g (Not the throat one). \\
\emp{e.g. :} \\
Lange = 'Langı' not 'Lanhgı' \\
\end{paracol}
\newpage
\subsection{To wear/To put on/To take off}
\subsubsection{To wear}
\begin{tabularx}{\textwidth}{ p{0.2\textwidth} p{0.3\textwidth} p{0.35\textwidth} }
 \hline
 Werkwoorden & Voorbeeld & Voor \\
 \hline
 Dragen & Ik \emp{draag} een pak. & Schoenen / Kleren / Sieraden \\
 Aanhebben & Ik \emp{heb} een pak \emp{aan}. & Schoenen / Kleren \\
 Omhebben & De docent \emp{heeft} een stropdas \emp{om}. & Stropdas / Ketting / Armband / Ring / Sjaal \\
 Ophebben & De jongen \emp{heeft} een muts \emp{op}. & Hoed / Bril \\
\end{tabularx}
\subsubsection{To put on}
\begin{tabularx}{\textwidth}{ p{0.2\textwidth} p{0.3\textwidth} p{0.35\textwidth} }
 \hline
 Werkwoorden & Voorbeeld & Voor \\
 \hline
 Omdoen & Zij \emp{doet} een sjaal \emp{om}. & Strapdos / Ketting / Armband / Ring / Sjaal \\
 Indoen & Zij \emp{doet} haar oorbellen \emp{in}. & Oorbellen \\
 Opzetten & Ik \emp{zet} mijn bril \emp{op}. & Bril / Hoed \\
 Aantrekken & Zij \emp{trekt} haar schoenen \emp{aan}. & Kleren / Schoenen \\
\end{tabularx}
\subsubsection{To take off}
\begin{tabularx}{\textwidth}{ p{0.2\textwidth} p{0.3\textwidth} p{0.35\textwidth} }
 \hline
 Werkwoorden & Voorbeeld & Voor \\
 \hline
 Afdoen & Jij \emp{doet} je ring \emp{af}. & Ketting / Armband / Ring / Sjaal \\
 Uitdoen & Zij \emp{doet} haar schoenen \emp{uit}. & Schoenen / Kleren / Oorbellen \\
 Afzetten & Hij \emp{zet} zijn hoed \emp{af}. & Bril / Hoed \\
 Uittrekken & Zij \emp{trekt} haar trui \emp{uit}. & Kleren \\
\end{tabularx}
\subsubsection{Opposites}
\begin{center}
\begin{tabularx}{180pt}{ c c c }
 \hline
 Omdoen & - & Afdoen \\
 Indoen & - & Uitdoen \\
 Opzetten & - & Afzetten \\
 Aantrekken & - & Uittrekken \\
\end{tabularx}
\end{center}
\newpage
\subsection{Grammatica}
\subsubsection{Het persoonlijk voornaamwoord}
\begin{tabularx}{\textwidth}{ p{0.2\textwidth} p{0.2\textwidth} p{0.2\textwidth} p{0.4\textwidth} }
 \hline
 Voornaamwoord & Persoonlijk voornaamwoord & Engels &  Voorbeeld \\
 \hline
 Ik & Mij / Me & Me & Peter ziet \emp{mij / me}. \\
 Je / Jij & Jou / Je & You & Peter ziet \emp{jou / je}. \\
 U & U & You & Peter ziet \emp{u}. \\
 Hij & Hem & Him & Peter ziet \emp{hem}. \\
 Ze / Zij & Haar & Her & Peter ziet \emp{haar}. \\
 Het & Het / Hij & It & Peter ziet \emp{het}. \\
 We / Wij & Ons & Us & Peter ziet \emp{ons}. \\
 Jullie & Jullie & You & Peter ziet \emp{jullie}. \\
 Ze / Zij & Hen / Ze & Them & Peter ziet \emp{hen / ze}. \\ 
\end{tabularx} \\ \\ \\
\attention{Important:} \\ \\
- For persons only both \emp{hen} and \emp{ze} can be used but for objects only \emp{ze} can be used. \\
- The rule for \emp{het/hij} as in \emp{it} is as follows: If it's a \emp{het} word then use \emp{het} otherwise use \emp{hij}. \\ \\
\emp{Extra examples:} \\ \\
- Ik heb een afspraak \underline{met} \emp{haar}. \\
- De boeken zijn \underline{van} \emp{jou}. \\
- Hij geeft bloemen \underline{aan} \emp{mij}. \\
- Ik heb informatie \underline{voor} \emp{u}. \\

\end{document}