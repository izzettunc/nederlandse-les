\documentclass[a4paper,14pt]{extarticle}
\usepackage{paracol}
\usepackage{tabularx}
\usepackage[a4paper, total={7in, 10.5in}]{geometry}
\newcommand{\optional}[1]{\underline{\textbf{(#1)}}\footnote{Optional}}
\newcommand{\note}[2]{\underline{\textbf{#1}}\footnote{#2}}
\newcommand{\attention}[1]{\underline{\textbf{!! #1}}}
\newcommand{\emp}[1]{\underline{\textbf{#1}}}
\begin{document}
\section{In de Winkel}
Het gaat over iets kopen en betalen. 
\subsection{Artikelen kopen en ruilen}
\begin{paracol}{2}
\begin{tabularx}{200pt}{ p{100pt} p{100pt} }
 \hline
 Engels & Nederlands \\
 \hline
 To return & Terugbrengen \\
 To exchange & Ruilen \\
 To fit & Passen \\
 Suit & Pak \\
 Dress & Jurk \\
 Gloves & Handschoenen \\
 Shoes & Schoenen \\
 Skirt & Rok \\
 Beanie & Muts \\
 Shirt & Overhemd \\
 Tie & Stropdas \\
 Socks & Sokken \\
 Boots & Laarzen \\
 Shawl & Sjaal \\
 Raincoat & Regenpak \\
 Clothing & Kleding \\
 Clothes & Kleren \\
 Pants & Broek \\
 Jeans & Spijkerbroek \\
 Hat & Hoed \\
 Cap & Pet \\
 Uniform & Uniform \\
\end{tabularx}
    \switchcolumn
\begin{tabularx}{200pt}{ p{100pt} p{100pt} }
 \hline
 Engels & Nederlands \\
 \hline
 Coat & Jas \\
 Shorter dress & Jurkje \\
 Sleveeless Undershirt & Hemd \\
 Fitting room & Paskamer \\
 Ankle boots & Enkellaarsjes \\
 Bracelet & Armband \\
 Necklace & Ketting \\
 Watch & Horloge \\
 Ring & Ring \\
 Shorts & Korte broek \\
 Jewelry & Sieraden \\
 Earrings & Oorbellen \\
\end{tabularx} \\ \\ \\
\emp{Het gaat over} \dots \emp{= It is about} \dots \\ \\
\attention{Pronounciation} \\ 
-ng is always pronounced with soft g (Not the throat one). \\
\emp{e.g. :} \\
Lange = 'Langı' not 'Lanhgı' \\
\end{paracol}
\subsection{To wear/To put on/To take off}
\subsubsection{To wear}
\begin{tabularx}{\textwidth}{ p{0.2\textwidth} p{0.3\textwidth} p{0.35\textwidth} }
 \hline
 Werkwoorden & Voorbeeld & Voor \\
 \hline
 Dragen & Ik \emp{draag} een pak. & Schoenen / Kleren / Sieraden \\
 Aanhebben & Ik \emp{heb} een pak \emp{aan}. & Schoenen / Kleren \\
 Omhebben & De docent \emp{heeft} een stropdas \emp{om}. & Stropdas / Ketting / Armband / Ring / Sjaal \\
 Ophebben & De jongen \emp{heeft} een muts \emp{op}. & Hoed / Bril \\
\end{tabularx}
\newpage
\subsubsection{To put on}
\begin{tabularx}{\textwidth}{ p{0.2\textwidth} p{0.3\textwidth} p{0.35\textwidth} }
 \hline
 Werkwoorden & Voorbeeld & Voor \\
 \hline
 Omdoen & Zij \emp{doet} een sjaal \emp{om}. & Strapdos / Ketting / Armband / Ring / Sjaal \\
 Indoen & Zij \emp{doet} haar oorbellen \emp{in}. & Oorbellen \\
 Opzetten & Ik \emp{zet} mijn bril \emp{op}. & Bril / Hoed \\
 Aantrekken & Zij \emp{trekt} haar schoenen \emp{aan}. & Kleren / Schoenen \\
\end{tabularx}
\subsubsection{To take off}
\begin{tabularx}{\textwidth}{ p{0.2\textwidth} p{0.3\textwidth} p{0.35\textwidth} }
 \hline
 Werkwoorden & Voorbeeld & Voor \\
 \hline
 Afdoen & Jij \emp{doet} je ring \emp{af}. & Ketting / Armband / Ring / Sjaal \\
 Uitdoen & Zij \emp{doet} haar schoenen \emp{uit}. & Schoenen / Kleren / Oorbellen \\
 Afzetten & Hij \emp{zet} zijn hoed \emp{af}. & Bril / Hoed \\
 Uittrekken & Zij \emp{trekt} haar trui \emp{uit}. & Kleren \\
\end{tabularx}
\subsubsection{Opposites}
\begin{center}
\begin{tabularx}{180pt}{ c c c }
 \hline
 Omdoen & - & Afdoen \\
 Indoen & - & Uitdoen \\
 Opzetten & - & Afzetten \\
 Aantrekken & - & Uittrekken \\
\end{tabularx}
\end{center}
\subsection{Grammatica}
\subsubsection{Het persoonlijk voornaamwoord}
\begin{tabularx}{\textwidth}{ p{0.2\textwidth} p{0.2\textwidth} p{0.2\textwidth} p{0.4\textwidth} }
 \hline
 Voornaamwoord & Persoonlijk voornaamwoord & Engels &  Voorbeeld \\
 \hline
 Ik & Mij / Me & Me & Peter ziet \emp{mij / me}. \\
 Je / Jij & Jou / Je & You & Peter ziet \emp{jou / je}. \\
 U & U & You & Peter ziet \emp{u}. \\
 Hij & Hem & Him & Peter ziet \emp{hem}. \\
 Ze / Zij & Haar & Her & Peter ziet \emp{haar}. \\
 Het & Het & It & Peter ziet \emp{het}. \\
 We / Wij & Ons & Us & Peter ziet \emp{ons}. \\
 Jullie & Jullie & You & Peter ziet \emp{jullie}. \\
 Ze / Zij & Hen / Ze & Them & Peter ziet \emp{hen / ze}. \\ 
\end{tabularx} 
\newpage
\hfill \\
\attention{Important:} \\ \\
- For persons both \emp{hen} and \emp{ze} can be used but for objects only \emp{ze} can be used. \\ \\
\emp{Extra examples:} \\ \\
\begin{tabularx}{\textwidth}{ p{0.4\textwidth} p{0.2\textwidth} p{0.35\textwidth}}
- \emp{De computer} werkt niet meer. & \emp{Hij} is kapot. & Ik breng \emp{hem} terug naar de winkel. \\
- Ik lees \emp{het boek}. & \emp{Het} is erg leuk. & Ik lees \emp{het} elke dag. \\
- Ik koop \emp{twee broeken}. & \emp{Ze} zijn blauw. & Ik vind \emp{ze} heel mooi. \\
\\\hline\\
- Ik heb een afspraak \underline{met} \emp{haar}. \\
- De boeken zijn \underline{van} \emp{jou}. \\
- Hij geeft bloemen \underline{aan} \emp{mij}. \\
- Ik heb informatie \underline{voor} \emp{u}. \\
\end{tabularx}
\subsubsection{Aanwijzende Woorden(Demonstrative pronouns)}
Used while pointing things.
\begin{center}
\begin{tabularx}{\textwidth}{p{0.15\textwidth} p{0.23\textwidth} p{0.23\textwidth} p{0.23\textwidth}}
 \hline
  & \textbf{Woord} & \note{Dichtbij / hier}{Close / here} & \note{Ver weg / daar}{Far away / there}\\
 \hline
 De-woord & De sok & Deze sok & Die sok \\
 Het-woord & Het vest & Dit vest & Dat sok \\
 \hline
  & \textbf{Tijd} & \note{Nu}{Now} & \note{Niet nu}{Not now}\\
 \hline
 De-woord & De dag& Deze dag & Die dag \\
 Het-woord & Het jaar & Dit jaar & Dat jaar \\
\end{tabularx}
\end{center}
\emp{e.g. :} \\\\
\emp{De sokken} zijn rood. - \emp{Deze sokken} zijn rood. \\
\emp{De jas} hangt aan \emp{de kapstok}. - \emp{Die jas} hangt aan \emp{die kapstok}. \\
\emp{Vorig jaar} een koude zomer gehad. - \emp{Dat jaar} een koude zomer gehad. \\
\subsubsection{Adjectief}
\begin{tabularx}{\textwidth}{ p{0.3\textwidth} p{0.3\textwidth} p{0.3\textwidth} }
 \hline
 \textbf{Rules} & \textbf{De-Woord} & \textbf{HET-Woord}  \\
 \hline \\
 If the word is specified with \emp{de} or \emp{het}, then adjective gets an \emp{"-e"}. & De klein\emp{e} fiets. \newline De mooi\emp{e} tuin. & Het klein\emp{e} meisje. \newline Het mooi\emp{e} huis. \newline Dat klein\emp{e} meisje. \newline Dat mooi\emp{e} huis. \newline Ons klein\emp{e} meisje. \newline Ons mooi\emp{e} huis. \\
 \\\hline\\
 If the word hasn't specified and therefore an \emp{"een"} used, then if only it's a \emp{"de"} word adjective gets an \emp{"-e"}. & Een klein\emp{e} fiets. \newline Een mooi\emp{e} tuin. & Een \emp{klein } meisje. \newline Een \emp{mooi } huis. \\
 \\\hline\\
 If the word is plural it's article changes to \emp{"de"} and adjective gets an \emp{"-e"}.  & De klein\emp{e} fietsen. \newline Klein\emp{e} fietsen. \newline De mooi\emp{e} tuinen. \newline Mooi\emp{e} tuinen. & De klein\emp{e} meisjes. \newline Klein\emp{e} meisjes. \newline De mooi\emp{e} huizen. \newline Mooi\emp{e} huizen. \\
 \\\hline\\
 If adjective is a material then no matter what condition the word is, it gets an \emp{"-en"}.  & De goud\emp{en} ring. \newline Een goud\emp{en} ring. \newline De goud\emp{en} ringen. & Het hout\emp{en} huis \newline Een hout\emp{en} huis. \newline De hout\emp{en} huizen. \\
 \\\hline\\
 If adjective is used as the object of the sentence then it stays in its base form. & De fiets is \emp{klein}. \newline Een fiets is \emp{klein}. \newline Fietsen zijn \emp{klein}. & Het huis is \emp{mooi}. \newline Een huis is \emp{mooi}. \newline Huizen zijn \emp{mooi}.
\end{tabularx} \\ \\ \\
\newpage
\hfill \\
\emp{Sommige adjectieven} \\
\begin{paracol}{2}
\hfill \\
\begin{tabularx}{200pt}{p{100pt} p{100pt}}
 \hline
 \textbf{Engels} & \textbf{Nederlands} \\
 \hline 
 \multicolumn{2}{ >{\centering}p{180pt} }{\note{Materialen}{Materials}} \\
 \hline
 Metal & Metaal \\
 Leather & Leer \\
 Wool & Wol \\
 Wood & Hout \\
 Gold & Goud \\
 Silver & Zilver \\
 Plastic & \note{Plastic}{As plastic is not a dutch word, when used it doesn't get an '-en'.} \\
 Paper & Papier \\
 Cotton & Katoen \\
 Iron & Ijzer \\
 Carton & Karton \\
 \hline 
 \multicolumn{2}{ >{\centering}p{180pt} }{\note{Vormen}{Shapes}} \\
 \hline
 Circle & Rond \\
 Square & Vierkant \\
 Triangle & Driehoek \\
 Rectangle & Rechthoek \\
 Oval & Ovaal \\
 \hline 
 \multicolumn{2}{ >{\centering}p{180pt} }{\note{Anderen}{Others}} \\
 \hline
 Beautiful & Mooi \\
 Difficult & Moeilijk \\
 Easy & Makkelijk \\
 Nice & Leuk \\
 Ugly & Lelijk \\
 Strong & Sterk \\
 Weak & Zwak \\
 Clean & Schoon \\
 Dirt & Vies \\
 Warm & Warm \\
 Cold & Koud \\
\end{tabularx}
\switchcolumn
\hfill \\
\begin{tabularx}{200pt}{p{100pt} p{100pt}}
 \hline
 \textbf{Engels} & \textbf{Nederlands} \\
 \hline 
 \multicolumn{2}{ >{\centering}p{180pt} }{\note{Maten}{Sizes}} \\
 \hline
 Big & Groot \\
 Small & Klein \\
 High & Hoog \\
 Low & Laag \\
 Fat & Dik \\
 Thin & Dun \\
 Tall & Lang \\
 Short & Kort \\
 Heavy & Zwaar \\
 Light & Licht \\
 \hline 
 \multicolumn{2}{ >{\centering}p{180pt} }{\note{Kleuren}{Colors}} \\
 \hline
 Green & Groen \\
 Red & Rood \\
 Blue & Blauw \\
 Yellow & Geel \\
 White & Wit \\
 Black & Zwart \\
 Orange & Oranje \\
 Pink & Roze \\
 Brown & Bruin \\
 Gray & Grijs \\
 Beige & Beige \\
 Purple & Paars \\
 Lilac & Lila \\
 Dark blue & Donkerblauw \\
 Light blue & Lichtblauw \\
 Turquoise & Turkoois \\
\end{tabularx}
\end{paracol}
\newpage
\subsubsection{Vergelijken(Comparison)}
\begin{tabularx}{\textwidth}{p{0.2\textwidth} p{0.8\textwidth}}
 \hline
\textbf{Type} & \textbf{Rule} \\
 \hline
Comparative & Item1 + zijn + (adjective + -er / -der) dan + Item2 \newline If word ends with \emp{'-r'} add \emp{'-der'}, otherwise add \emp{'-der'}.\\\\
Superlative & Item1 + zijn + het + (adjective + st). \\
\end{tabularx} \\ \\ \\
\attention{Examples:} \\
\begin{center}
\begin{tabularx}{400pt}{p{75pt} p{100pt} p{100pt} p{100pt}}
\hline
& Jas optie 1 & Jas optie 2 & Jas optie 3\\
\hline
Maat & L & M & S \\
Prijs & 10 Euro & 20 Euro & 50 Euro \\
Nieuwheid & Oud & Niuew & Normaal \\
Kleuren & Zwart & Grijs & Wit \\
\end{tabularx} 
\end{center}
- De zwarte jas is \emp{groeter dan} de witte jas. \\
- De zwarte jas is \emp{goedkoper dan} de grijze jas. \\
- De witte jas is \emp{duurder dan} de grijze jas. \\
- De zwarte jas is \emp{ouder dan} de witte jas. \\
- De zwarte jas is \emp{donkerder} dan de grijze jas. \\
- De zwarte jas is \emp{het goedkopest}. \\
- De grijze jas is \emp{het nieuwst}. \\
- De witte jas is \emp{het kleinst}. \\
- Je bent het best. \\
- Ik eet \emp{graag} pasta maar ik eet \emp{liever} pizza en ik eet \emp{het liefts} friet. \\ \\
\emp{Even ... als en net zo ... als} \\ \\
\begin{tabularx}{\textwidth}{p{0.2\textwidth} p{0.8\textwidth}}
 \hline
\textbf{Type} & \textbf{Rule} \\
 \hline
Even \dots als \newline Net zo \dots als &  Item1 + zijn + \emp{even / net zo} + adjective + \emp{als} + Item2\\
\end{tabularx} \\ \\ 
\attention{Examples:} \\
\begin{center}
\begin{tabularx}{400pt}{p{75pt} p{100pt} p{100pt} p{100pt}}
\hline
& Jas optie 1 & Jas optie 2 & Jas optie 3\\
\hline
Maat & L & M & M \\
Prijs & 20 Euro & 20 Euro & 50 Euro \\
Nieuwheid & Oud & Niuew & Oud \\
Kleuren & Zwart & Grijs & Wit \\
\end{tabularx} 
\end{center}
- De zwarte jas is even duur als de grijse jas. \\
- De grijze jas is net zo groet als de witte jas. \\
- De witte jas is even oud als de zwarte jas. \\
- Hij is net zo snel als een \note{paard}{Horse}.
\subsubsection{Er met voorzetsel}
In some cases er can be combined with prepositons like "uit", "mee", "in", "door", "op", etc\dots \\
\begin{center}
\begin{tabularx}{0.7\textwidth}{p{0.15\textwidth} p{0.2\textwidth} p{0.15\textwidth} p{0.2\textwidth}}
 \hline
 \textbf{Woord} & \textbf{Voorbeeld} & \textbf{Analyse} & \textbf{Engels}\\
 \hline
 De pen & Ik schrijf ermee. & er + mee & with it \\
 De vork & Ik eet ermee. & er + mee & with it \\
 Het glas & Ik drink eruit & er + uit & out of it \\
 De fiets & Ik rijd erop & er + op & on it \\
 De bril & Ik kijk erdoor & er + door & through it \\
 Het boek & Ik lees erin & er + in & in it \\
\end{tabularx}
\end{center}
\end{document}