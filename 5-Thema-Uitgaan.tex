\documentclass[a4paper,14pt]{extarticle}
\usepackage{paracol}
\usepackage{tabularx}
\usepackage[a4paper, total={7in, 10.5in}]{geometry}
\newcommand{\optional}[1]{\underline{\textbf{(#1)}}\footnote{Optional}}
\newcommand{\note}[2]{\underline{\textbf{#1}}\footnote{#2}}
\newcommand{\attention}[1]{\underline{\textbf{!! #1}}}
\newcommand{\emp}[1]{\underline{\textbf{#1}}}
\begin{document}
\section{Uitgaan}
\begin{paracol}{2}
\begin{tabularx}{200pt}{p{100pt} p{100pt}}
    \hline
    \textbf{Engels} & \textbf{Nederlands} \\
    \hline
    Soup bowl & De Soepkom \\
    Wine bottle & De wijnfles \\
    Candle & De kaars \\
    Pan  & De pan \\
    Fork & De vork \\
    Candlestick & De kandelaar \\
    Wine glass & Het wijnglas \\
    Coaster & De onderzetter \\
    Bowl & De schaal \\
    Glass & Het glas \\
    Spoon & De lepel \\
    Napkin & Het servet \\
    Plate & Het bord \\
    Knife & Het mes \\
    Lid & Het deksel \\
    \note{Menu}{choices} & Het menu \\
    \note{Menu}{the card that menu has written on} & De menukaart \\
\end{tabularx}
\switchcolumn
\emp{Notes:} \\
Uitgaan = To go out for an \note{activity}{like dinner, cinema, bar, etc.} \\
Uit eten gaan = To eat out \\
Naar buiten gaan = To go \note{outside}{has a literal meaning, so probably for a walk.} \\ \\
Meegaan = Come along, go along, go with \\ \\
Het hangt ervan af = It depends \\ 
Het hangt af van mijn stemming = It depends on my mood \\
Dat hangt ervan af of ik heb tijd = That depends if I have time \\ \\
In het midden van de tafel = In the middle of the table \\
\end{paracol}
\subsection{Grammatica}
\subsubsection{Er als plaats}
Another way of using 'er" is as a substitute for places.\\ \\
\begin{tabularx}{\textwidth}{p{0.4\textwidth} p{0.3\textwidth} p{0.3\textwidth}}
    \hline
    \textbf{Normaal zin} & \textbf{Zin met 'er'} & \textbf{'er' als} \\
    \hline
   \emp{Amsterdam} is een leuke stad. & Ben jij \emp{er} geweest? & Amsterdam \\
   Ik ga naar \emp{de supermarkt}. & Ga jij \emp{er} naartoe? & De supermarkt \\
   Ik ga volgende week naar \emp{Utrecht}. & Ik ga \emp{er} volgende week naartoe. & Utrecht \\
   Ik kom \note{vaak}{Often} bij \emp{de sportschool} & Ik kom \emp{er} vaak. & De sportschool \\
\end{tabularx}
\newpage
\subsubsection{Want en maar}
Want(as in because) and Maar(as in but) used to connect sentences together to give more related information. \\ 
\hfill \\
\attention{Rule:} \\ \\
- Sentence1 + want + \note{Sentence2}{that indicates a reason}. \\
- Sentence1 + maar + \note{Sentence2}{that indicates a contrast}. \\ \\
\begin{tabularx}{\textwidth}{p{0.35\textwidth} >{\centering}p{0.25\textwidth} p{0.4\textwidth}}
    \hline
    \textbf{Eerste zin} & \textbf{Want / Maar} & \textbf{Tweede zin} \\
    \hline
    \note{Ik kan niet naar de film}{gaan is omitted}, & want & ik moet werken. \\
    Ik kan mandaag niet, & maar & dinsdag heb ik \note{wel}{emphasize} tijd. \\
    Je moet een tafel reserveren, & want & het restaurant is \note{heel}{very} populier. \\
    Ik wil een huis kopen, & maar & ik heb geen geld. \\
    Rita gaat niet mee, & want & ze heeft geen  tijd. \\
\end{tabularx}
\subsubsection{De voltooide tijd(Past tense)}
This tense always needs an auxiliary verb. These verbs are 'Zijn' and 'Hebben'. \\ \\
\attention{Rule: } 
\begin{itemize}
\item Subject + hebben / zijn + rest of the sentence + verb in voltooide form. 
    \begin{description}
        \item[Voltooide form:] \hfill   
        \begin{itemize}
        \item \emp{ge + verb in ik form + d or t}
            \begin{description}
            \item[Except:] the verb can be irregular an in this case, there is no rule.
            \item[Except:] the verb starts with \emp{'-ge'}, \emp{'-be'}, \emp{'-ver'}, \emp{'-her'}, \emp{'-ont'}. \\ In these cases it is only: \emp{verb in ik form + d or t}  
            \item[d or t:] You can decide if you need to add \emp{-d} or \emp{-t} at the end of the verb using \emp{SOFTKETCHUP} rule.
                \begin{itemize}
                \item If verb ends with \emp{S,F,T,K,C,H,P} then add \emp{-t}
                \item else add \emp{-d}
                    \begin{description}
                    \item[Except:] If the last letter of word in \emp{ik} form is different from the original form. Then you need to decide based on the original form.
                    \item[Except:] If the ik form already has a \emp{-t} or \emp{-d} at the end.
                    \end{description}
                \end{itemize}
            \end{description}
        \end{itemize}
    \end{description}
\end{itemize}
\newpage
\hfill \\
\attention{Examples for voltooide form: } \\ \\ \\
\begin{tabularx}{\textwidth}{p{0.15\textwidth} p{0.15\textwidth} p{0.15\textwidth} p{0.45\textwidth}}
    \hline
    \textbf{Original} & \textbf{Ik form} & \textbf{Voltooide} & \textbf{Explanation} \\
    \hline
    Werken & Wer\emp{k} & \emp{Ge}werk\emp{t} & \emp{K} is in \emp{S, F, T, K, C, H, P}. \\
    Wonen & Woo\emp{n} & \emp{Ge}woon\emp{d} & \emp{N} is \emp{not} in \emp{S, F, T, K, C, H, P}. \\
    Maken & Maa\emp{k} & \emp{Ge}maak\emp{t} & \emp{K} is in \emp{S, F, T, K, C, H, P}. \\
    Bellen & Be\emp{l} & \emp{Ge}bel\emp{d} & \emp{L} is \emp{not} in \emp{S, F, T, K, C, H, P}. \\
    Proeven & Proef & \emp{Ge}proef\emp{d} & \emp{V} in original verb has changed to \emp{F} therefore, we apply SOFTKETCHUP rule onto \emp{V} which is \emp{not} in \emp{S, F, T, K, C, H, P}. \\
    Reizen & Reis & \emp{Ge}reis\emp{d} & \emp{Z} in original verb has changed to \emp{S} therefore, we apply SOFTKETCHUP rule onto \emp{Z} which is \emp{not} in \emp{S, F, T, K, C, H, P}. \\
    Gebruiken & \emp{Ge}bruik\emp{t} & \emp{Gebruikt} & No change as it already starts with \emp{-ge} and ends with \emp{-t}. \\
    Herhalen & \emp{Her}haa\emp{l} & \emp{Her}haal\emp{d} & No \emp{-ge} as it starts with \emp{-her} and \emp{L} is \emp{not} in \emp{S, F, T, K, C, H, P}. \\
    Zijn & Ben & \emp{Geweest} & Irregular verb, no rule. \\
    Beginnen & Begin & \emp{Begonnen} & Irregular verb, no rule. \\
\end{tabularx} \\ \\ \\ \\
\attention{Examples for sentences: } \\ \\ \\
\begin{tabularx}{\textwidth}{p{0.45\textwidth} p{0.45\textwidth}}
    \hline
    \textbf{Normaal} & \textbf{Voltooide tijd} \\
    \hline
    Zij \emp{danst} de \note{hele}{whole} avond. & Zij \emp{heeft} de hele avond \emp{gedanst}. \\
    Ik \emp{kook} voor haar \note{gezin}{family}. & Ik \emp{heb} voor haar gezin \emp{gekookt}. \\
    Ik \emp{drink} koffie. & Ik \emp{heb} koffie \emp{gedronken}. \\
    Ik \emp{ben} in Italie. & Ik \emp{ben} in Italie \emp{geweest}. \\
    Ik \emp{ga} naar Amsterdam. & Ik \emp{ben} naar Amsterdam \emp{gegaan}. \\
    Het glas \emp{\note{valt}{to fall}} van de tafel. & Het glas \emp{is} van de tafel \emp{gevallen}. \\
\end{tabularx} \\ \\ \\
\end{document}